To maximise the impact and long-term sustainability of Bonsai.ML, this proposal
pursues seven aims:  

\begin{enumerate}

  \item \textbf{Documentation} – Produce comprehensive, user-centred
documentation that makes ML tools accessible to non-specialists across the
neuroscience community.  

  \item \textbf{Training} – Develop and deliver a practical training course on
Bonsai and Bonsai.ML, building capacity and lowering barriers to adoption.  

  \item \textbf{Dissemination} – Publish the first Bonsai.ML paper to increase
      visibility and uptake within the scientific community.

  \item \textbf{Maintainability} – In collaboration with Microsoft Research
  Cambridge, integrate their C\# probabilistic programming library
  \emph{Infer.NET} into Bonsai.ML. This will simplify and unify inference and
  learning code, making it faster, more maintainable, and more extensible,
  while embedding the expertise of a world-leading industrial research group
  into the Bonsai ecosystem.

  \item \textbf{Community reach} – Engage neuroscientists interested in
\emph{closed-loop} neural experimentation by supporting the use of
\href{https://cloctools.github.io/}{CLOCTools} within Bonsai.ML, in
collaboration with Prof.~Garrett Stanley (Georgia Tech). This partnership will
attract a new community of researchers to Bonsai and broaden its reach to an
emerging but underrepresented area of neuroscience.

  \item \textbf{Community building} – Strengthen the developer community by
organising the second Bonsai Developers Conference in December 2026, building
on the successful inaugural event in 2024.  

  \item \textbf{Governance} – Establish a steering committee to guide a
long-term roadmap, prioritise sustainability, and supervise development
throughout the grant.

\end{enumerate}

By the end of the funding period, Bonsai.ML will provide high-quality
documentation and training resources, a robust and maintainable codebase, and a
stronger developer community with expanded expertise and broader reach. In
particular, collaborations with Microsoft Research Cambridge and
Prof.~Garrett Stanley’s laboratory will embed cutting-edge knowledge in
probabilistic inference and closed-loop experimentation. Together, these
efforts will ensure Bonsai remains a sustainable, widely adopted research
software platform, aligned with UKRI’s strategic priorities in artificial
intelligence, bioscience, and community-driven research software
sustainability.

\subsubsection*{Measures of impact}

We are currently using the measures described below to assess the impact of
Bonsai in the experimental neuroscience and/or methods development community.
We will compare the change of these measures before and after each project
milestone is achieved.

\begin{description}

    \item[Bonsai.ML nuget package downloads:] we monitor the number of
    Bonsai.ML package downloads at
    \url{https://www.nuget.org/packages/Bonsai.ML},

    \item[Number of Bonsai packages integrated into Bonsai.ML:] we use
    nuget.org to check the number of packages that are using Bonsai.ML at
    \url{https://www.nuget.org/packages/Bonsai.ML}, which is a proxy to the
    number of methods developers contributing to Bonsai.ML.

    \item[Bonsai.ML discussions in Bonsai forum:] we track discussions in the
    Bonsai forum that are related to Bonsai.ML (e.g.,
    \url{https://github.com/orgs/bonsai-rx/discussions?discussions_q=is%3Aopen+bonsai.ml}.

    \item[Usage of \url{https://github.com/bonsai-rx/machinelearning}:] we
    observe the stars/watching/clones of this repository

    \item[Citations to the forthcoming Bonsai.ML paper.]

\end{description}
