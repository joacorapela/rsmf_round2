
\subsubsection*{Motivation}

Bonsai\footnote[1]{\url{https://bonsai-rx.org/}} is a software ecosystem used by thousands
of (mostly experimental neuroscience) users in the UK and all around the world
(7,000 downloads per year and 1,000 citations per year of the core Bonsai
paper~\citep{lopesEtAl15}).
%
Being a visual-programming language, Bonsai allows scientists with little
programming experience to control sophisticated neuroscience experiments.

Machine learning (ML) is now essential for most branches of science,
neuroscience in particular.
%
Yet, current neuroscience experiments are still controlled by simple means
(e.g., deliver a reward when a rat pokes left but not right).
%
In 2022, we realised that adding ML functionality to Bonsai could empower
Bonsai users and enable a radically new type of intelligent experimental
control.
%
We created the
Bonsai.ML\footnote[2]{\url{https://bonsai-rx.org/machinelearning}}
package providing machine learning functionality to the Bonsai ecosystem.

Since most experimental neuroscientists currently using Bonsai are not highly
skilled in ML, to help experimental neuroscientists adopt machine learning
tools and maximise the impact of Bonsai.ML, we need to invest extra
efforts on documentation, training and community building, as we propose below.

\subsubsection*{Relevance to the UK Research Ecosystem}

Bonsai.rx has established itself as mission-critical infrastructure within the
UK’s world-leading neuroscience and behavioral research communities. Its
adoption is widespread across premier institutions; at
\href{https://www.ucl.ac.uk/}{University College London (UCL)} alone, it is
utilized by more than ten research groups at the
\href{https://www.sainsburywellcome.org/}{Sainsbury Wellcome Centre}, as well
as the \href{https://www.ucl.ac.uk/brain-sciences/cortexlab}{Cortex Lab}, the
\href{https://coen-lab.com/}{Coen Lab}, and the
\href{https://www.ibn.ucl.ac.uk/}{Institute for Behavioural Neuroscience}.
Notably, \href{https://www.saleemlab.com/}{Prof. Aman Saleem} at UCL developed
\href{https://bonvision.github.io/info/Home/}{BonVision}, a globally recognized
Bonsai-based package for high-performance virtual reality environments.

The framework's reach extends to other national centers of excellence,
including the \href{https://www.tripodilab.org/}{Kinetic Cognition Lab} at the
\href{https://www2.mrc-lmb.cam.ac.uk/}{MRC Laboratory of Molecular Biology} and
the \href{https://www.crick.ac.uk/research/labs/petr-znamenskiy}{Znamenskiy
Lab} at the \href{https://www.crick.ac.uk/}{Francis Crick Institute}. In these
settings, Bonsai is not merely an auxiliary tool but the backbone of
experimental control and data acquisition; its discontinuation would cause
severe disruption to active, high-priority UK research programs.

Consolidating the machine learning functionality of Bonsai will substantially
expand the power of experimental tools available to these groups and lead to
new scientific discoveries in the UK.

\subsubsection*{Democratization of ML}

A core objective of this project is the democratization of machine learning
(ML) within the UK’s experimental neuroscience community. By leveraging
Bonsai.ML, we provide a low-code, reactive interface that empowers
experimentalists to integrate sophisticated inference models directly into
their experimental loops. This aligns with UKRI’s high-priority mandate to
upskill the biological workforce in data science and AI.

Training experimental scientists in ML is no longer a luxury but a fundamental
requirement for the next generation of research. Specifically, Bonsai.ML is
enabling:

\begin{enumerate}

    \item\textbf{Active Learning \& Real-time Closed-loop Control}: Moving
    beyond post-hoc analysis by allowing scientists to use ML to trigger
    experimental interventions based on real-time neural state detection.

    \item\textbf{Model Interpretability}: Ensuring that those closest to the
    biological data understand the assumptions and limitations of the ML models
    they employ, leading to more robust and reproducible science.

    \item\textbf{Research Software Sustainability}: Reducing the reliance on
    `black-box' commercial software by fostering a cohort of
    `Scientist-Developers' who can maintain and extend open-source tools within
    the UK infrastructure.

\end{enumerate}

\subsubsection*{Relation to current experimental control software ecosystem}
The large field of technologies serving experimental control and behaviour
monitoring is traditionally occupied either by domain-specific graphical user
interfaces for control and recording of specific devices and experiment types
(e.g.\ Open Ephys GUI\footnote[16]{\url{open-ephys.org/gui/}}, Miniscope DAQ
Software\footnote[17]{\url{github.com/Aharoni-Lab/Miniscope-DAQ-QT-Software}})
or by real-time control frameworks for specifying task logic using state
machine or similar formalisms (e.g. NIMH
ML\footnote[18]{\url{monkeylogic.nimh.nih.gov/}},
pyControl\footnote[19]{\url{pycontrol.readthedocs.io/en/latest/}},
Autopilot\footnote[20]{\url{docs.auto-pi-lot.com/en/latest/}},
Sanworks\footnote[21]{\url{sanworks.io/index.php}}).
%
These dedicated interfaces are typically very comfortable for experimenters in
the specific domain for which the tool is designed, but can become unwieldy
with the introduction of a new device or task variation from outside their
usual scope.
%
Alternatively, one can use a more general programming language such as Python
or MATLAB, with the disadvantage of the code being harder to create,
understand, maintain, and change.

Programming languages like LabVIEW straddle the middle ground and provide a
high-level, flexible visual interface for composing data acquisition and
control systems. Unlike Bonsai, however, the graphical elements of LabVIEW are
heterogeneous and very fine grained, thus requiring long and complex logical
structures to implement even a simple experimental control system.
%
A severe disadvantage of LabVIEW for the research community is that it is not
free nor open source.

By providing an extremely simple, yet flexible visual syntax, Bonsai provides
the opportunity for even complete non-programmers to design and successfully
customise relatively complex experiments from the ground up. It is this
capability in particular which has made Bonsai such an attractive standard tool
in experimental neuroscience.
%
In addition, Bonsai is free and open source.
