
\begin{center}
    \begin{tabular}{| l | l | l |}\hline
        \multicolumn{1}{|c|}{\textbf{Name}} &
        \multicolumn{1}{c|}{\textbf{Role}} &
        \multicolumn{1}{c|}{\textbf{Organization}}\\\hline\hline
        Prof.~Thomas Mrsic-Flogel (TMF) & Project lead & Sainsbury Wellcome Centre\\\hline
        Prof.~Maneesh Sahani (MS) & Project co-lead & Gatsby Unit\\\hline
        Dr.~Joaqu\'{i}n Rapela (JR) & Research software engineer & Gatsby Unit\\\hline
        Dr.~Gon\c{c}alo Lopes (GL) & Project co-lead & NeuroGEARS Ltd\\\hline
        Dr.~Nicholas Guilbeault (NG) & Research software engineer & NeuroGEARS Ltd\\\hline
    \end{tabular}
\end{center}

\subsubsection{Skills, experience and connection to software}

\paragraph{TMF} is an experimental neuroscientists and director of the SWC.
Research in his laboratory aims to explain how the brain makes decisions by
combining sensory information with previously learned knowledge. As the
behavioural tasks used in his lab require complex software-control of data
acquisi- tion and data analysis pipelines, he knows first-hand their crucial
importance for driving and enabling neuroscientific research.

He has published 49 peer-reviewed papers, with an h-index of 35 (calculated by
google scholar).  He is a founding member of the International Brain Laboratory
(IBL et al 2019 Neuron). The Bonsai ecosystem is critical to IBL, as it ensures
that experimental control, stimulus presentation and data acquisition can be
identically reproduced across all participating labs in UK, Europe and USA (IBL
et al 2021, eLife).

TMF was the project lead in the BBSRC project that funded the creation of
Bonsai.ML.

\paragraph{MS} is a computational neuroscientist and director of the Gatsby
Unit. He has authored over 150 peer-reviewed scientific papers, with an h-index
(computed by Google scholar) of 57. A substantial component of his research
focuses on the development of advanced machine-learning tools for
neuroscience research.

Beginning around 2005, his group published a series of new neuroinformatics
tools designed to characterise and understand population-scale activity using
the large-scale multielectrode recording methods. These papers
provided the backbone for a new analytic approach that is now being employed
and extended by systems neuroscience laboratories worldwide.
%
A central component of the current proposal is to disseminate this approach
(and others) already available within Bonsai, easing its adoption by a wider
group of laboratories that lack in-house informatics expertise.

MS was the project co-lead in the BBSRC project that funded the creation of
Bonsai.ML.

\paragraph{GL} is the creator of Bonsai and director of NeuroGEARS Ltd, a
non-profit company that leads the development of Bonsai. He has ample
experience in software engineering, holding a Licentiate degree in Computer
Science from NOVA University Lisbon, and having worked between 2006 and 2010 at
the NOVA CENTRIA Artificial Intelligence laboratory, and at YDreams, where he
was leading a team developing a ground-breaking engine for Augmented Reality.

Transitioning into his neuroscience PhD at the Champalimaud Foundation, he
created the Bonsai visual programming language to run his PhD experiments,
which then led him to managing a software development company serving thousands
of users and collaborating with leading universities and research centres
around the world.

NeuroGEARS was the project partner in the BBSRC project that funded the
creation of Bonsai.ML.

\paragraph{JR} specialises in
signal processing and machine learning, with applications to understanding
brain function (Rapela et al., 2006, Rapela et al., 2010, Rapela et al., 2018
and Rapela et al., 2019).

He has extensive software development expertise, holding a Master’s degree in
Computer Science and industry experience at IBM Argentina and the IBM Almaden
Research Center, US.
%
He joined the Gatsby Computational Neuroscience Unit in 2019 as a Research
Engineer Fellow.
%
He is the lead developer of \href{https://github.com/joacorapela/svGPFA}{Sparse
Variational Gaussian Process Factor Analysis (svGPFA)}, and has openly released
several \href{https://github.com/joacorapela}{other probabilistic machine
learning packages}.

JR played a leading role in securing the BBSRC grant that funded the
creation of Bonsai.ML and has led its development since the project’s
inception. He prepared the current proposal, but could not act as project lead
due to UCL regulations.

\paragraph{NG} is a research software engineer with expertise in real-time
machine learning, neural data analysis, and open-source software development.
He holds a PhD in neuroscience from the University of Toronto, where he
developed the \href{https://ncguilbeault.github.io/BonZeb/}{BonZeb} software
for zebrafish kinematic tracking, closed-loop stimulation, and neural data
analysis.

From 2023 to 2026 he was a research software engineer at the Gatsby
Computational Neuroscience Unit and he recently joined NeuoGEARS as a software
engineer. He is the core developer of the
\href{https://bonsai-rx.org/machinelearning}{Bonsai.ML} project, integrating
machine learning methods into the Bonsai visual reactive programming language.

\subsubsection{Balance of Skills and Expertise}

Our team has the required expertise, at the leadership and development levels,
in machine learning (MS, JR, NG), software development (GL, JR, NG), neuroscience
(TMF, MS, GL, JR, NG) and experimental control (GL, NG).

Our expertise is complemented by that of world-class project partners in
close-loop neural control (Prof.~Garrett Stanley), probabilistic programming
(Dr.~Tom Minka), high-channel-count electrophysiological recordings (Dr.~Josh
Siegle) and vision and navigation (Prof.~Aman Saleem).  Please refer to their
letters of support.

\subsubsection{Community engagement}

Bonsai is built on a decentralized development model.  While the core engine is
maintained by NeuroGEARS, over 70% of the 130+ available packages are
contributed by external research groups. Prominent examples include BonVision
(UCL) and Bonsai.Harp (Champalimaud Foundation), which have become essential
infrastructure for the global neuroscience community. This `distributed
maintenance' model ensures that the software evolves at the pace of scientific
discovery and is not vulnerable to a single point of failure (e.g., a single
lab or company).

Several channels exist to engage Bonsai users in the development process.
%
First, Bonsai has a \href{https://github.com/orgs/bonsai-rx/discussions}{Forum}
where users and Bonsai experts interact on a daily fashion to troubleshoot
problems and share their experience.
%
Second, there is a weekly Bonsai Dev Club, open to the public, where Bonsai developers
meet once a week to discuss Bonsai development issues (notes from these
meetings appear
\href{https://github.com/orgs/bonsai-rx/discussions/categories/dev-club}{here})..
%
Third, NeuroGEARS has been delivering two or more
\href{https://bonsai-rx.org/learn/}{Bonsai courses} every year at
major universities, as the Champalimaud Center for the Unknown or the
Sainsbury Wellcome Centre. In these meetings Bonsai developers and users
interact closely and build long-lasting collaborations.
%
Fourth, Bonsai users can join the bi-annual Bonsai Developer Conference,
where Bonsai users and developers meet for a one-week-long presentations about
recent Bonsai developments and to discuss the Bonsai development plan.
The \href{https://conference.bonsai-rx.org/2024/}{First Bonsai Developer
Conference} took place at the SWC in December, 2024, and was a total success
hosting more than 30 developers from across the globe. The Second Bonsai
Developers Conference will take place on December 2026.

\subsubsection{Development practices}

Bonsai is an open-source platform built with robust software engineering
practices, including modular design, automated testing, semantic versioning,
and comprehensive documentation.  Sustained by public/private funding and the
Bonsai Foundation CIC, Bonsai promotes transparency, reproducibility, and
long-term sustainability in neuroscience software.

