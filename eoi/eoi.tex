\documentclass[12pt]{article}

\usepackage[colorlinks=true]{hyperref}
\usepackage{xcolor}
\usepackage{natbib}
\usepackage[margin=1.5cm]{geometry}

\usepackage{tikz}

\usepackage[most]{tcolorbox}

\makeatletter
\newcommand*{\radiobutton}{%
  \@ifstar{\@radiobutton0}{\@radiobutton1}%
}
\newcommand*{\@radiobutton}[1]{%
  \begin{tikzpicture}
    \pgfmathsetlengthmacro\radius{height("X")/2}
    \draw[radius=\radius] circle;
    \ifcase#1 \fill[radius=.6*\radius] circle;\fi
  \end{tikzpicture}%
}
\makeatother

\usepackage{enumitem,amssymb}
\newlist{todolist}{itemize}{2}
\setlist[todolist]{label=$\square$}
\usepackage{pifont}
\newcommand{\cmark}{\ding{51}}%
\newcommand{\xmark}{\ding{55}}%
\newcommand{\done}{\rlap{$\square$}{\raisebox{2pt}{\large\hspace{1pt}\cmark}}%
\hspace{-2.5pt}}
\newcommand{\wontfix}{\rlap{$\square$}{\large\hspace{1pt}\xmark}}

\title{Letter of Intent for \href{https://www.software.ac.uk/research-software-maintenance-fund/round-2}{Research Software Maintenance Fund -- Round 2}}
\author{}

\newenvironment{instruction}{%
    \begin{tcolorbox}[breakable,colback=red!5,colframe=red,title=Instruction]%
	}{%
    	\end{tcolorbox}%
	}

\begin{document}
\maketitle
\tableofcontents

\section{Start here}

\subsection{Category}

RSMF Round 2

\subsection{Application name}

Consolidating Bonsai as a Standard for Neuroscience Intelligent Experimental Control

\subsection{Research Area}

Biological, Medical and Veterinary Sciences

\pagebreak

\section{Elegibility}

\begin{instruction}

    This criterion considers whether the proposed work is in scope for and
    eligible for the Research Software Maintenance Fund (RSMF). The RSMF aims
    to fund work that will improve the sustainability of research software,
    particularly improving maintenance, reducing technical debt, improving user
    experience, and building community. The RSMF is especially interested in
    funding work for which it is hard to find funding elsewhere.

\end{instruction}

\subsection{Software to be maintained}

Bonsai

\subsection{Programming language}

C\#

\subsection{Software maturity and UK research benefit}

\begin{instruction}

    500 words

    Please demonstrate how the software(s) you maintain:

    \begin{itemize}

        \item Supports research being carried out in the United Kingdom.

        \item Is beyond the prototype/pilot stage and has had multiple stable /
            production releases (i.e. versions that have been released to end
            users, not just used for testing or evaluation).

        \item Is used by people outside their organisation(s) and is used
            beyond the immediate collaborators of the software’s
            developers/wider project team.

        \item Has mechanisms for raising issues, providing feedback, and adding
            additional contributors.

    \end{itemize}

\end{instruction}

\subsection{Support to UK research}

Bonsai is a visual reactive programming that can and has been used in a wide variety of
domains, such as artistic performances and virtual guides to museums. However,
in recent years the development of Bonsai has focused on experimental
neuroscience.

Currently Bonsai is used at multiple neuroscience research laboratories.  Just
to mention a few, it is used at the by more than ten research groups at the
Sainsbury Welcome Centre; it is used at the Institute for Behavioural
Neuroscience where Prof.  Aman Saleem created the popular BonVision package; it
is also used at the Cortex Lab and Coen Lab. All the previous research group
are part of University College London (UCL). Beyond UCL, Bonsai is used by the Kinetic
Cognition Lab at the MRC Laboratory of Molecular Biology, and at the Znamenskiy
Lab at the Francis Crick Institute.


\subsection{Lead organization}

\begin{instruction}
    This is the legal organisation employing the Project Lead.

    The organisation needs to be \href{}{eligible for UKRI funding}.
\end{instruction}

\pagebreak

\section{Vision and impact}

\subsection{Motivation, Vision and Impact}

\begin{instruction}
    500 words

    \begin{itemize}

        \item The applicants demonstrate a clear need in the research community,
            a clear benefit to UK research, and the resulting impact on the
            research that is possible.

        \item The applicants understand the current software ecosystem (including
            other software used in this research area, and how this differs
            from their software) and researcher use cases / workflows (the way
            that software is used in combination) for the research areas their
            software targets.

        \item The vision of the proposed work is clearly articulated.

    \end{itemize}

\end{instruction}


\subsubsection*{Motivation}

Bonsai\footnote[1]{\url{https://bonsai-rx.org/}} is a software ecosystem used by thousands
of (mostly experimental neuroscience) users in the UK and all around the world
(7,000 downloads per year and 1,000 citations per year of the core Bonsai
paper~\citep{lopesEtAl15}).
%
Being a visual-programming language, Bonsai allows scientists with little
programming experience to control sophisticated neuroscience experiments.

Machine learning (ML) is now essential for most branches of science,
neuroscience in particular.
%
Yet, current neuroscience experiments are still controlled by simple means
(e.g., deliver a reward when a rat pokes left but not right).
%
In 2022, we realised that adding ML functionality to Bonsai could empower
Bonsai users and enable a radically new type of intelligent experimental
control.
%
We created the
Bonsai.ML\footnote[2]{\url{https://bonsai-rx.org/machinelearning}}
package providing machine learning functionality to the Bonsai ecosystem.

Since most experimental neuroscientists currently using Bonsai are not highly
skilled in ML, to help experimental neuroscientists adopt machine learning
tools and maximise the impact of Bonsai.ML, we need to invest extra
efforts on documentation, training and community building, as we propose below.

\subsubsection*{Relevance to the UK Research Ecosystem}

Bonsai.rx has established itself as mission-critical infrastructure within the
UK’s world-leading neuroscience and behavioral research communities. Its
adoption is widespread across premier institutions; at
\href{https://www.ucl.ac.uk/}{University College London (UCL)} alone, it is
utilized by more than ten research groups at the
\href{https://www.sainsburywellcome.org/}{Sainsbury Wellcome Centre}, as well
as the \href{https://www.ucl.ac.uk/brain-sciences/cortexlab}{Cortex Lab}, the
\href{https://coen-lab.com/}{Coen Lab}, and the
\href{https://www.ibn.ucl.ac.uk/}{Institute for Behavioural Neuroscience}.
Notably, \href{https://www.saleemlab.com/}{Prof. Aman Saleem} at UCL developed
\href{https://bonvision.github.io/info/Home/}{BonVision}, a globally recognized
Bonsai-based package for high-performance virtual reality environments.

The framework's reach extends to other national centers of excellence,
including the \href{https://www.tripodilab.org/}{Kinetic Cognition Lab} at the
\href{https://www2.mrc-lmb.cam.ac.uk/}{MRC Laboratory of Molecular Biology} and
the \href{https://www.crick.ac.uk/research/labs/petr-znamenskiy}{Znamenskiy
Lab} at the \href{https://www.crick.ac.uk/}{Francis Crick Institute}. In these
settings, Bonsai is not merely an auxiliary tool but the backbone of
experimental control and data acquisition; its discontinuation would cause
severe disruption to active, high-priority UK research programs.

Consolidating the machine learning functionality of Bonsai will substantially
expand the power of experimental tools available to these groups and lead to
new scientific discoveries in the UK.

\subsubsection*{Democratization of ML}

A core objective of this project is the democratization of machine learning
(ML) within the UK’s experimental neuroscience community. By leveraging
Bonsai.ML, we provide a low-code, reactive interface that empowers
experimentalists to integrate sophisticated inference models directly into
their experimental loops. This aligns with UKRI’s high-priority mandate to
upskill the biological workforce in data science and AI.

Training experimental scientists in ML is no longer a luxury but a fundamental
requirement for the next generation of research. Specifically, Bonsai.ML is
enabling:

\begin{enumerate}

    \item\textbf{Active Learning \& Real-time Closed-loop Control}: Moving
    beyond post-hoc analysis by allowing scientists to use ML to trigger
    experimental interventions based on real-time neural state detection.

    \item\textbf{Model Interpretability}: Ensuring that those closest to the
    biological data understand the assumptions and limitations of the ML models
    they employ, leading to more robust and reproducible science.

    \item\textbf{Research Software Sustainability}: Reducing the reliance on
    `black-box' commercial software by fostering a cohort of
    `Scientist-Developers' who can maintain and extend open-source tools within
    the UK infrastructure.

\end{enumerate}

\subsubsection*{Relation to current experimental control software ecosystem}
The large field of technologies serving experimental control and behaviour
monitoring is traditionally occupied either by domain-specific graphical user
interfaces for control and recording of specific devices and experiment types
(e.g.\ Open Ephys GUI\footnote[16]{\url{open-ephys.org/gui/}}, Miniscope DAQ
Software\footnote[17]{\url{github.com/Aharoni-Lab/Miniscope-DAQ-QT-Software}})
or by real-time control frameworks for specifying task logic using state
machine or similar formalisms (e.g. NIMH
ML\footnote[18]{\url{monkeylogic.nimh.nih.gov/}},
pyControl\footnote[19]{\url{pycontrol.readthedocs.io/en/latest/}},
Autopilot\footnote[20]{\url{docs.auto-pi-lot.com/en/latest/}},
Sanworks\footnote[21]{\url{sanworks.io/index.php}}).
%
These dedicated interfaces are typically very comfortable for experimenters in
the specific domain for which the tool is designed, but can become unwieldy
with the introduction of a new device or task variation from outside their
usual scope.
%
Alternatively, one can use a more general programming language such as Python
or MATLAB, with the disadvantage of the code being harder to create,
understand, maintain, and change.

Programming languages like LabVIEW straddle the middle ground and provide a
high-level, flexible visual interface for composing data acquisition and
control systems. Unlike Bonsai, however, the graphical elements of LabVIEW are
heterogeneous and very fine grained, thus requiring long and complex logical
structures to implement even a simple experimental control system.
%
A severe disadvantage of LabVIEW for the research community is that it is not
free nor open source.

By providing an extremely simple, yet flexible visual syntax, Bonsai provides
the opportunity for even complete non-programmers to design and successfully
customise relatively complex experiments from the ground up. It is this
capability in particular which has made Bonsai such an attractive standard tool
in experimental neuroscience.
%
In addition, Bonsai is free and open source.


\subsection{Benefits}

\begin{instruction}

    250 words

    \begin{itemize}

        \item The benefit to the long-term maintenance, governance, and
            adoption of the software is clearly demonstrated.

        \item The applicants demonstrate that the project will lead to
            advancement of good research software practices appropriate for
            their area / community.

        \item The project has considered how to increase the variety of users
            and contributors (for instance, the number of different
            organisations where users / contributors are based; the career
            stages of contributors).

    \end{itemize}

\end{instruction}

The proposed documentation and dissemination activities will help the users of
Bonsai.ML better understand the ML methods in the package, perform more
sophisticated neuroscience experiments, and produce unprecedented new
neuroscientific findings.
%
In addition, these activities will attract to Bonsai new experimental
neuroscience users interested in adding ML functionality to their experiments,
and create a new Bonsai community of machine learning methods developers, as we
explain below.
%
We have focused this proposal on applications of Bonsai.ML to experimental
neuroscience, since this is our area of expertise. However, Bonsai is used in
other experimental domains, like live exhibitions and robotics, where Bonsai.ML
should also be relevant.

Bonsai has demonstrated that providing experimental neuroscientists easy to use
tools for experimental control, allows them to create very sophisticated
experiments.
%
Bonsai.ML puts ML tools in the hands of experimental neuroscientists. With
these new tools and documentation, the level of sophistication of their
experiments, and the research findings that they produce, should greatly increase.

Most current ML methods are designed to operate offline, with datasets stored
on disk, after data collection has finished.
%
Bonsai requires ML methods that can process online data, while data is being
collected, and in a close-loop manner.
%
This has two important implications.
%
First, Bonsai.ML provides a new type of ML methods for real-time neuroscience
data, operating in in close loop. Used by experimental neuroscientists, these
method could generate unprecedented findings on brain function.
%
Second, Bonsai, as an excellent source of real-time neural and behavioural data,
could become of interest to ML methods developers wanting to apply their
methods to real-time data in close loop.
%
Therefore, the dissemination of Bonsai.ML among machine learning methods
developers could attract to Bonsai a new community of ML methods developers.


\subsection{EDIA}

\begin{instruction}

    250 words

    \begin{itemize}

        \item The project has considered the accessibility of the software and
            explains how this could be improved

        \item The project understands their current position in relation to
            equity, diversity, and inclusivity and explains how this could be
            improved.

    \end{itemize}

\end{instruction}

We are committed to ensuring that Bonsai.ML is accessible, inclusive, and 
beneficial to the widest possible community of neuroscience researchers.

The Bonsai.ML team itself is diverse, with leadership from Croatia, India, 
and Portugal, and research software engineers from Argentina and Canada. 
This diversity shapes our perspective and makes us strong advocates of EDIA. 
Bonsai already has a global user base spanning all five continents. It 
empowers researchers in disadvantaged communities, notably in Eastern Europe 
and South America. A recent example is the Transatlantic Behavioral 
Neuroscience School (Argentina, August 2025), where Bonsai was used to enable 
cutting-edge training opportunities across borders.

At its core, Bonsai embodies inclusivity: it allows non-programmers to design 
and run sophisticated experiments. This lowers barriers for researchers from 
underrepresented groups, smaller institutions, or disciplines outside computer 
science---broadening participation in methods that would otherwise remain the 
preserve of elite or technically specialized groups. With Bonsai.ML, we extend 
this philosophy by providing non-programmers with state-of-the-art machine 
learning tools.

The activities supported by this grant will further embed EDIA principles. 
Comprehensive documentation and training will lower entry barriers through 
step-by-step tutorials, plain-language explanations, captioned video materials, 
and diverse experimental examples. All materials will be openly available and 
designed with accessibility in mind (e.g.\ screen-reader compatibility, 
colourblind-friendly figures). Our documentation approach will closely follow 
the \texttt{scikit-learn} project, which is internationally recognized for 
embracing EDIA principles through clarity, consistency, and accessible 
contribution pathways.

Finally, community events such as the 2026 Bonsai Developers Conference will 
adopt a code of conduct, select a diverse range of speakers across career 
stages, genders, and regions, and provide hybrid participation options to
reduce 
financial and travel barriers.



\pagebreak

\section{Feasibility and Approach}

\begin{instruction}

This criterion considers the way that the proposed work will be carried out.
Here we look for evidence that the funded work has a clear plan that will
deliver the objectives and benefits.

While we do not expect a budget at this stage of the application process,
please do keep in mind that the maximum we will fund is £150k for a duration of
one year.

\end{instruction}

\subsection{Proposed work}

\begin{instruction}

    500 words

    \begin{itemize}

        \item The objectives, activities and outputs of the work are clearly
            described and relate to each other and to the benefits.

        \item The project team have considered how the way they measure the
            progress of their project will be applied to this work.

    \end{itemize}

\end{instruction}

To maximise the impact and long-term sustainability of Bonsai.ML, this proposal
pursues seven aims:  

\begin{enumerate}

  \item \textbf{Documentation} – Produce comprehensive, user-centred
documentation that makes ML tools accessible to non-specialists across the
neuroscience community.  

  \item \textbf{Training} – Develop and deliver a practical training course on
Bonsai and Bonsai.ML, building capacity and lowering barriers to adoption.  

  \item \textbf{Dissemination} – Publish the first Bonsai.ML paper to increase
      visibility and uptake within the scientific community.

  \item \textbf{Maintainability} – In collaboration with Microsoft Research
  Cambridge, integrate their C\# probabilistic programming library
  \emph{Infer.NET} into Bonsai.ML. This will simplify and unify inference and
  learning code, making it faster, more maintainable, and more extensible,
  while embedding the expertise of a world-leading industrial research group
  into the Bonsai ecosystem.

  \item \textbf{Community reach} – Engage neuroscientists interested in
\emph{closed-loop} neural experimentation by supporting the use of
\href{https://cloctools.github.io/}{CLOCTools} within Bonsai.ML, in
collaboration with Prof.~Garrett Stanley (Georgia Tech). This partnership will
attract a new community of researchers to Bonsai and broaden its reach to an
emerging but underrepresented area of neuroscience.

  \item \textbf{Community building} – Strengthen the developer community by
organising the second Bonsai Developers Conference in December 2026, building
on the successful inaugural event in 2024.  

  \item \textbf{Governance} – Establish a steering committee to guide a
long-term roadmap, prioritise sustainability, and supervise development
throughout the grant.

\end{enumerate}

By the end of the funding period, Bonsai.ML will provide high-quality
documentation and training resources, a robust and maintainable codebase, and a
stronger developer community with expanded expertise and broader reach. In
particular, collaborations with Microsoft Research Cambridge and
Prof.~Garrett Stanley’s laboratory will embed cutting-edge knowledge in
probabilistic inference and closed-loop experimentation. Together, these
efforts will ensure Bonsai remains a sustainable, widely adopted research
software platform, aligned with UKRI’s strategic priorities in artificial
intelligence, bioscience, and community-driven research software
sustainability.

\subsubsection*{Measures of impact}

We are currently using the measures described below to assess the impact of
Bonsai in the experimental neuroscience and/or methods development community.
We will compare the change of these measures before and after each project
milestone is achieved.

\begin{description}

    \item[Bonsai.ML nuget package downloads:] we monitor the number of
    Bonsai.ML package downloads at
    \url{https://www.nuget.org/packages/Bonsai.ML},

    \item[Number of Bonsai packages integrated into Bonsai.ML:] we use
    nuget.org to check the number of packages that are using Bonsai.ML at
    \url{https://www.nuget.org/packages/Bonsai.ML}, which is a proxy to the
    number of methods developers contributing to Bonsai.ML.

    \item[Bonsai.ML discussions in Bonsai forum:] we track discussions in the
    Bonsai forum that are related to Bonsai.ML (e.g.,
    \url{https://github.com/orgs/bonsai-rx/discussions?discussions_q=is%3Aopen+bonsai.ml}.

    \item[Usage of \url{https://github.com/bonsai-rx/machinelearning}:] we
    observe the stars/watching/clones of this repository

    \item[Citations to the forthcoming Bonsai.ML paper.]

\end{description}


\subsection{Software sustainability}

\begin{instruction}

    250 words

    \begin{itemize}

        \item The future sustainability of the software (i.e. the software will
            continue to be available in the future, on new platforms, meeting
            new needs) has been considered, and the project has developed, or
            is committed to developing, a roadmap (a summary of the vision for
            the software and the direction of the software over time), a
            governance structure, and a sustainability/business plan.

    \end{itemize}

\end{instruction}

Bonsai.ML will be sustained beyond the RSMF funding period through a
combination of community, institutional, and commercial support.
%
NeuroGEARS, which already underpins the wider Bonsai ecosystem, invests a fixed
proportion of its service income into Bonsai maintenance and will extend this
support to Bonsai.ML.

Our academic partners are also strongly invested.
%
The Sainsbury Wellcome Centre
(SWC) relies on Bonsai for experimental control and Bonsai.ML for advanced
control in some of its experiments, and contributes financially to its
development.
%
The Gatsby Unit will continue to provide machine learning
expertise to guide Bonsai.ML’s growth.
%
Prof. Stanley’s lab will contribute its expertise in the control of
physiological signals to extend Bonsai.ML’s capabilities.

Finally, our collaboration with Microsoft Research Cambridge ensures that 
Infer.NET integration brings sustained industrial expertise to Bonsai.ML, 
strengthening its long-term sustainability beyond the funding period.



\pagebreak

\section{Capability to Deliver}

\begin{instruction}

    This criterion considers the capability of the team, processes, and
    procedures in place for maintaining and developing software, and delivering
    the proposed work.

\end{instruction}

\subsection{Delivery team and mechanisms}

\begin{instruction}

    500 words

    \begin{itemize}

        \item The Leads / Co-Leads have appropriate skills, experience and
            connection to the software to achieve the objectives of the work.

        \item There are clear mechanisms for engaging the community and seeking
            external contributions and feedback, and how this will be
            incorporated into the work.

        \item Approaches to development of the software being used are suitable
            for the software and achieving the goals of the work.

    \end{itemize}

\end{instruction}


\begin{center}
    \begin{tabular}{| l | l | l |}\hline
        \multicolumn{1}{|c|}{\textbf{Name}} &
        \multicolumn{1}{c|}{\textbf{Role}} &
        \multicolumn{1}{c|}{\textbf{Organization}}\\\hline\hline
        Prof.~Thomas Mrsic-Flogel (TMF) & Project lead & Sainsbury Wellcome Centre\\\hline
        Prof.~Maneesh Sahani (MS) & Project co-lead & Gatsby Unit\\\hline
        Dr.~Joaqu\'{i}n Rapela (JR) & Research software engineer & Gatsby Unit\\\hline
        Dr.~Gon\c{c}alo Lopes (GL) & Project co-lead & NeuroGEARS Ltd\\\hline
        Dr.~Nicholas Guilbeault (NG) & Research software engineer & NeuroGEARS Ltd\\\hline
    \end{tabular}
\end{center}

\subsubsection{Skills, experience and connection to software}

\paragraph{TMF} is an experimental neuroscientists and director of the SWC.
Research in his laboratory aims to explain how the brain makes decisions by
combining sensory information with previously learned knowledge. As the
behavioural tasks used in his lab require complex software-control of data
acquisi- tion and data analysis pipelines, he knows first-hand their crucial
importance for driving and enabling neuroscientific research.

He has published 49 peer-reviewed papers, with an h-index of 35 (calculated by
google scholar).  He is a founding member of the International Brain Laboratory
(IBL et al 2019 Neuron). The Bonsai ecosystem is critical to IBL, as it ensures
that experimental control, stimulus presentation and data acquisition can be
identically reproduced across all participating labs in UK, Europe and USA (IBL
et al 2021, eLife).

TMF was the project lead in the BBSRC project that funded the creation of
Bonsai.ML.

\paragraph{MS} is a computational neuroscientist and director of the Gatsby
Unit. He has authored over 150 peer-reviewed scientific papers, with an h-index
(computed by Google scholar) of 57. A substantial component of his research
focuses on the development of advanced machine-learning tools for
neuroscience research.

Beginning around 2005, his group published a series of new neuroinformatics
tools designed to characterise and understand population-scale activity using
the large-scale multielectrode recording methods. These papers
provided the backbone for a new analytic approach that is now being employed
and extended by systems neuroscience laboratories worldwide.
%
A central component of the current proposal is to disseminate this approach
(and others) already available within Bonsai, easing its adoption by a wider
group of laboratories that lack in-house informatics expertise.

MS was the project co-lead in the BBSRC project that funded the creation of
Bonsai.ML.

\paragraph{GL} is the creator of Bonsai and director of NeuroGEARS Ltd, a
non-profit company that leads the development of Bonsai. He has ample
experience in software engineering, holding a Licentiate degree in Computer
Science from NOVA University Lisbon, and having worked between 2006 and 2010 at
the NOVA CENTRIA Artificial Intelligence laboratory, and at YDreams, where he
was leading a team developing a ground-breaking engine for Augmented Reality.

Transitioning into his neuroscience PhD at the Champalimaud Foundation, he
created the Bonsai visual programming language to run his PhD experiments,
which then led him to managing a software development company serving thousands
of users and collaborating with leading universities and research centres
around the world.

NeuroGEARS was the project partner in the BBSRC project that funded the
creation of Bonsai.ML.

\paragraph{JR} specialises in
signal processing and machine learning, with applications to understanding
brain function (Rapela et al., 2006, Rapela et al., 2010, Rapela et al., 2018
and Rapela et al., 2019).

He has extensive software development expertise, holding a Master’s degree in
Computer Science and industry experience at IBM Argentina and the IBM Almaden
Research Center, US.
%
He joined the Gatsby Computational Neuroscience Unit in 2019 as a Research
Engineer Fellow.
%
He is the lead developer of \href{https://github.com/joacorapela/svGPFA}{Sparse
Variational Gaussian Process Factor Analysis (svGPFA)}, and has openly released
several \href{https://github.com/joacorapela}{other probabilistic machine
learning packages}.

JR played a leading role in securing the BBSRC grant that funded the
creation of Bonsai.ML and has led its development since the project’s
inception. He prepared the current proposal, but could not act as project lead
due to UCL regulations.

\paragraph{NG} is a research software engineer with expertise in real-time
machine learning, neural data analysis, and open-source software development.
He holds a PhD in neuroscience from the University of Toronto, where he
developed the \href{https://ncguilbeault.github.io/BonZeb/}{BonZeb} software
for zebrafish kinematic tracking, closed-loop stimulation, and neural data
analysis.

From 2023 to 2026 he was a research software engineer at the Gatsby
Computational Neuroscience Unit and he recently joined NeuoGEARS as a software
engineer. He is the core developer of the
\href{https://bonsai-rx.org/machinelearning}{Bonsai.ML} project, integrating
machine learning methods into the Bonsai visual reactive programming language.

\subsubsection{Balance of Skills and Expertise}

Our team has the required expertise, at the leadership and development levels,
in machine learning (MS, JR, NG), software development (GL, JR, NG), neuroscience
(TMF, MS, GL, JR, NG) and experimental control (GL, NG).

Our expertise is complemented by that of world-class project partners in
close-loop neural control (Prof.~Garrett Stanley), probabilistic programming
(Dr.~Tom Minka), high-channel-count electrophysiological recordings (Dr.~Josh
Siegle) and vision and navigation (Prof.~Aman Saleem).  Please refer to their
letters of support.

\subsubsection{Community engagement}

Bonsai is built on a decentralized development model.  While the core engine is
maintained by NeuroGEARS, over 70% of the 130+ available packages are
contributed by external research groups. Prominent examples include BonVision
(UCL) and Bonsai.Harp (Champalimaud Foundation), which have become essential
infrastructure for the global neuroscience community. This `distributed
maintenance' model ensures that the software evolves at the pace of scientific
discovery and is not vulnerable to a single point of failure (e.g., a single
lab or company).

Several channels exist to engage Bonsai users in the development process.
%
First, Bonsai has a \href{https://github.com/orgs/bonsai-rx/discussions}{Forum}
where users and Bonsai experts interact on a daily fashion to troubleshoot
problems and share their experience.
%
Second, there is a weekly Bonsai Dev Club, open to the public, where Bonsai developers
meet once a week to discuss Bonsai development issues (notes from these
meetings appear
\href{https://github.com/orgs/bonsai-rx/discussions/categories/dev-club}{here})..
%
Third, NeuroGEARS has been delivering two or more
\href{https://bonsai-rx.org/learn/}{Bonsai courses} every year at
major universities, as the Champalimaud Center for the Unknown or the
Sainsbury Wellcome Centre. In these meetings Bonsai developers and users
interact closely and build long-lasting collaborations.
%
Fourth, Bonsai users can join the bi-annual Bonsai Developer Conference,
where Bonsai users and developers meet for a one-week-long presentations about
recent Bonsai developments and to discuss the Bonsai development plan.
The \href{https://conference.bonsai-rx.org/2024/}{First Bonsai Developer
Conference} took place at the SWC in December, 2024, and was a total success
hosting more than 30 developers from across the globe. The Second Bonsai
Developers Conference will take place on December 2026.

\subsubsection{Development practices}

Bonsai is an open-source platform built with robust software engineering
practices, including modular design, automated testing, semantic versioning,
and comprehensive documentation.  Sustained by public/private funding and the
Bonsai Foundation CIC, Bonsai promotes transparency, reproducibility, and
long-term sustainability in neuroscience software.



\bibliographystyle{apalike}
\bibliography{bonsai}

\end{document}
